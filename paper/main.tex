\documentclass[12pt]{article}
%\documentclass{article}
\usepackage[utf8]{inputenc}
\usepackage[margin=1.0in]{geometry}
\usepackage[version=4]{mhchem}
\usepackage{siunitx}
\usepackage{amsmath}
\usepackage{amssymb}
\usepackage{graphicx}
% \usepackage{pgfplots}
\usepackage[skip=10pt plus1pt, indent=20pt]{parskip}

\usepackage{bibentry}

\DeclareSIUnit\gauss{G}
\DeclareSIUnit\erg{erg}

\title{\vspace{-2em} {\bf Observation of Eclipsing Binaries}}
\author{Matt Ketkaroonkul}
\date{June 6, 2023}

\DeclareSIUnit\day{d}
\begin{document}

\maketitle

\section{Abstract}

\section{Data and Observations}
Quantifying light curves for eclipsing binaries is an essential task in the field of astrophysics, enabling the detailed study of these fascinating systems. Eclipsing binaries consist of two stars orbiting each other in close proximity, periodically passing in front of each other, leading to a characteristic variation in their observed brightness over time. By accurately measuring and analyzing these light curves, astronomers can derive fundamental parameters of the binary system, such as the stellar masses, radii, and orbital properties.
We observed two such star systems: V* TZ Boo and V* CC Com. We gather light curve data of each star system for time series analysis to compare with existing data of the eclipsing binaries. Given our observing time constraints with the APO ARCSAT, we picked the eclipsing binary systems for their short orbital period of around 0.3 d. For instance, V* TZ Boo has a period of 0.297 d and V* CC Com has a period of 0.221 d.
We determine the orbital period of the eclipsing binaries by using curve fitting methods. Methods include Markov Chain Monte-Carlo (MCMC) and Lomb-Scargle periodograms. From determining the orbital motions of the eclipsing binaries, we are able to constrain the masses of the constituent stars. This paper will also outline the statistical uncertainties and limitations of our observation project. 

\section{Methods}
\section{Discussion of Results}
\section{Conclusion}
\section{Acknowledgements}
\end{document}
